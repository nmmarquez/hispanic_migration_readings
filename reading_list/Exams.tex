\documentclass[11pt,]{article}
\usepackage[margin=1in]{geometry}
\newcommand*{\authorfont}{\fontfamily{phv}\selectfont}
\usepackage[]{mathpazo}
\usepackage{abstract}
\renewcommand{\abstractname}{}    % clear the title
\renewcommand{\absnamepos}{empty} % originally center
\newcommand{\blankline}{\quad\pagebreak[2]}

\providecommand{\tightlist}{%
  \setlength{\itemsep}{0pt}\setlength{\parskip}{0pt}} 
\usepackage{longtable,booktabs}

\usepackage{parskip}
\usepackage{titlesec}
\titlespacing\section{0pt}{12pt plus 4pt minus 2pt}{6pt plus 2pt minus 2pt}
\titlespacing\subsection{0pt}{12pt plus 4pt minus 2pt}{6pt plus 2pt minus 2pt}

\titleformat*{\subsubsection}{\normalsize\itshape}

\usepackage{titling}
\setlength{\droptitle}{-.25cm}

%\setlength{\parindent}{0pt}
%\setlength{\parskip}{6pt plus 2pt minus 1pt}
%\setlength{\emergencystretch}{3em}  % prevent overfull lines 

\usepackage[T1]{fontenc}
\usepackage[utf8]{inputenc}

\usepackage{fancyhdr}
\pagestyle{fancy}
\usepackage{lastpage}
\renewcommand{\headrulewidth}{0.3pt}
\renewcommand{\footrulewidth}{0.0pt} 
\lhead{}
\chead{}
\rhead{\footnotesize Comprehensive Exam Reading List and
Rationale -- Winter 2020}
\lfoot{}
\cfoot{\small \thepage/\pageref*{LastPage}}
\rfoot{}

\fancypagestyle{firststyle}
{
\renewcommand{\headrulewidth}{0pt}%
   \fancyhf{}
   \fancyfoot[C]{\small \thepage/\pageref*{LastPage}}
}

%\def\labelitemi{--}
%\usepackage{enumitem}
%\setitemize[0]{leftmargin=25pt}
%\setenumerate[0]{leftmargin=25pt}




\makeatletter
\@ifpackageloaded{hyperref}{}{%
\ifxetex
  \usepackage[setpagesize=false, % page size defined by xetex
              unicode=false, % unicode breaks when used with xetex
              xetex]{hyperref}
\else
  \usepackage[unicode=true]{hyperref}
\fi
}
\@ifpackageloaded{color}{
    \PassOptionsToPackage{usenames,dvipsnames}{color}
}{%
    \usepackage[usenames,dvipsnames]{color}
}
\makeatother
\hypersetup{breaklinks=true,
            bookmarks=true,
            pdfauthor={ ()},
             pdfkeywords = {},  
            pdftitle={Comprehensive Exam Reading List and Rationale},
            colorlinks=true,
            citecolor=blue,
            urlcolor=blue,
            linkcolor=magenta,
            pdfborder={0 0 0}}
\urlstyle{same}  % don't use monospace font for urls


\setcounter{secnumdepth}{0}





\usepackage{setspace}

\title{Comprehensive Exam Reading List and Rationale}
\author{Neal Marquez}
\date{Winter 2020}


\begin{document}  

		\maketitle
		
	
		\thispagestyle{firststyle}

%	\thispagestyle{empty}


	\noindent \begin{tabular*}{\textwidth}{ @{\extracolsep{\fill}} lr @{\extracolsep{\fill}}}


E-mail: \texttt{\href{mailto:nmarquez@u.washington.edu}{\nolinkurl{nmarquez@u.washington.edu}}} & Web: TBD\\
Office Hours: TBD  &  Class Hours: TBD\\
Office: TBD  & Class Room: TBD\\
	&  \\
	\hline
	\end{tabular*}
	
\vspace{2mm}
	


\hypertarget{description}{%
\section{Description}\label{description}}

Historically Latin America, and particularly Mexico, has been a large
source of the overall immigration that has occurred in the United
States. The relatively young population of Latin American countries and
the demand for low wage labor in the United States have acted as push
and pull factors for migration respectively. Geographic distribution,
however, has widely varied for immigration events. In addition to
traditional immigrant hubs Latin American immigrants into the United
States have founded large populations On the Southern Border region and
Chicago, in the first wave of immigration events, and at the turn of
century more towards South and South Eastern States.

Most recently, migration from Mexico, which has the largest single
foreign born population in the United States, has come to a tapering
off, while migration from other Latin American countries, such as
Central American countries, have begun to pick up. Despite this,
populations of foreign born Mexicans continue to migrate within the
United States shifting the locations where traditional migrant hubs were
found. The continued migration of the Mexican foreig born population has
not well been theorized upon and, as this population ages, we would
expect the push and pull factors for migration events to evolve from
predominantly labor concerns, to potentially housing, affordability, and
health concerns.

The following readings cover general demographic concepts, as well as
migration and migration from Latin America to the United States. Taking
those readings, I aim to build a theoretical framework for analyzing
migration data of the Mexican born population living in the United
States and why we see different patterns of internal migration relative
to other populations. Strong understanding of how core demographic
patterns may alter the internal migration process is essential for
understanding the specifics related to the Mexican migration process.
This work expands upon the traditional economic factors that accompany
studies of internal migration and focuses both on the receiving and
sending context.

\hypertarget{reading-list}{%
\section{Reading List}\label{reading-list}}

\hypertarget{demography}{%
\subsection{Demography}\label{demography}}

\hypertarget{overview-and-evolution}{%
\subsubsection{Overview and Evolution}\label{overview-and-evolution}}

Bongaarts, John (1978). ``A Framework for Analyzing the Proximate
Determinants of Fertility'\,'. In:
\emph{Population and Development Review} 4.1, p.~105.

Caldwell, J.C. (1996). ``Demography and Social Science'\,'. In:
\emph{Population Studies} 50.3, pp.~305--333.

Caldwell, John C. (1990). ``Cultural and Social Factors Influencing
Mortality Levels in Developing Countries'\,'. In:
\emph{The ANNALS of the American Academy of Political and Social Science}
510.1, pp.~44--59.

Coale, Ansley J. (1975). ``The demographic transition'\,'. In:
\emph{The Population Debate: Dimensions and Perspectives} 1, pp.~53--72.

Greenhalgh, Susan (2009). ``The Social Construction of Population
Science: An Intellectual, Institutional, and Political History of
Twentieth-Century Demography'\,'. In:
\emph{Comparative Studies in Society and History} 38.01, p.~26.

Greg J. Duncan, Greg j. (2008). ``When to Promote, and When to Avoid, a
Population Perspective'\,'. In: \emph{Demography} 45.4, pp.~763--784.

Hirschman, Charles (1994). ``Why Fertility Changes'\,'. In:
\emph{Annual Review of Sociology} 20.1, pp.~203--233.

Lee, Ronald (2003). ``The Demographic Transition: Three Centuries of
Fundamental Change'\,'. In: \emph{Journal of Economic Perspectives}
17.4, pp.~167--190.

Moffitt, Robert (2005). ``Remarks on the analysis of causal
relationships in population research.'\,'. In: \emph{Demography} 42.1,
pp.~91--108.

Preston, Samuel H. (1993). ``The Contours of Demography: Estimates and
Projections'\,'. In: \emph{Demography} 30.4, p.~593.

-------- (1996). ``Population Studies of Mortality'\,'. In:
\emph{Population Studies} 50.3, pp.~525--536.

Tienda, Marta (2002). ``Demography and the Social Contract'\,'. In:
\emph{Demography} 39.4, pp.~587--616.

Xie, Yu (2000). ``Demography: Past, Present, and Future'\,'. In:
\emph{Journal of the American Statistical Association} 95.450, p.~670.

\hypertarget{social-determinants-and-inequality}{%
\subsubsection{Social Determinants and
Inequality}\label{social-determinants-and-inequality}}

Chetty, Raj, Nathaniel Hendren, Maggie Jones, and Sonya Porter (2018).
\emph{Race and Economic Opportunity in the United States: An Intergenerational Perspective}.
Cambridge, MA: National Bureau of Economic Research.

Chetty, Raj, Nathaniel Hendren, and Lawrence F. Katz (2016). ``The
Effects of Exposure to Better Neighborhoods on Children: New Evidence
from the Moving to Opportunity Experiment'\,'. In:
\emph{American Economic Review} 106.4, pp.~855--902.

Hayward, Mark D. and Bridget K. Gorman (2004). ``The Long Arm of
Childhood: The Influence of Early-Life Social Conditions on Men's
Mortality'\,'. In: \emph{Demography} 41.1, pp.~87--107.

Link, B. G. and J. Phelan (1995).
\emph{Social conditions as fundamental causes of disease.}.

Preston, Samuel H. (1975). ``The changing relation between mortality and
level of economic development. 1975.'\,'. In: \emph{Population Studies}
29.2, pp.~231--248.

Preston, Samuel H, Mark E. Hill, and Greg L. Drevenstedt (1998).
``Childhood conditions that predict survival to advanced ages among
African--Americans'\,'. In: \emph{Social Science \& Medicine} 47.9,
pp.~1231--1246.

Read, Jen'nan Ghazal and Bridget K. Gorman (2006). ``Gender inequalities
in US adult health: The interplay of race and ethnicity'\,'. In:
\emph{Social Science \& Medicine} 62.5, pp.~1045--1065.

Rogers, Richard G, Bethany G Everett, Jarron M Saint Onge, and Patrick M
Krueger (2010). ``Social, behavioral, and biological factors, and sex
differences in mortality.'\,'. In: \emph{Demography} 47.3, pp.~555--78.

Sampson, Robert J and Patrick Sharkey (2008). ``Neighborhood selection
and the social reproduction of concentrated racial inequality.'\,'. In:
\emph{Demography} 45.1, pp.~1--29.

Singh-Manoux, Archana and Michael Marmot (2005). ``Role of socialization
in explaining social inequalities in health'\,'. In:
\emph{Social Science \& Medicine} 60.9, pp.~2129--2133.

Szreter, Simon (1988). ``The Importance of Social Intervention in
Britain's Mortality Decline c .1850--1914: a Re-interpretation of the
Role of Public Health'\,'. In: \emph{Social History of Medicine} 1.1,
pp.~1--38.

\hypertarget{methods-and-data}{%
\subsubsection{Methods and Data}\label{methods-and-data}}

Bhroilchain, Maire Ni and Tim Dyson (2007). ``On Causation in
Demography: Issues and Illustrations'\,'. In:
\emph{Population and Development Review} 33.1, pp.~1--36.

Caldwell, John C. (1985).
\emph{Strengths and limitations of the survey approach for measuring and understanding fertility change: alternative possibilities.}.

Hobcraft, J, J Menken, and S Preston (1982). ``Age, period, and cohort
effects in demography: a review.'\,'. In: \emph{Population index} 48.1,
pp.~4--43.

Maine, Deborah, Lynn Freedman, Farida Shaheed, and Schuyler Frautschi
(1994). ``Risk, reproduction, and rights: The uses of reproductive
health data'\,'. In: \emph{Reproductive Health Matters} 3.6, pp.~40--51.

Palmore, James A. and Robert W. Gardner (1994).
\emph{Measuring mortality, fertility, and natural increase : a self-teaching guide to elementary measures}.
East-West Center, p.~169.

Preston, Sam, Patrick Heuveline, and Michel Guillot (2000).
\emph{Demography: Measuring and Modeling Population Processes}.
Wiley-Blackwell.

Raftery, Adrian E, Leontine Alkema, and Patrick Gerland (2014).
``Bayesian Population Projections for the United Nations.'\,'. In:
\emph{Statistical science : a review journal of the Institute of Mathematical Statistics}
29.1, pp.~58--68.

Snipp, C. Matthew (2003). ``Racial Measurement in the American Census:
Past Practices and Implications for the Future'\,'. In:
\emph{Annual Review of Sociology} 29.1, pp.~563--588.

Wachter, Kenneth W. (2014). \emph{Essential demographic methods}. 1st
ed. Harvard University Press, p.~288.

Wakefield, Jon, Geir-Arne Fuglstad, Andrea Riebler, Jessica Godwin,
Katie Wilson, et al. (2018). ``Estimating under-five mortality in space
and time in a developing world context'\,'. In:
\emph{Statistical Methods in Medical Research}, p.~096228021876798.

Wang, Haidong, Laura Dwyer-Lindgren, Katherine T Lofgren, Julie Knoll
Rajaratnam, Jacob R Marcus, et al. (2012). ``Age-specific and
sex-specific mortality in 187 countries, 1970--2010: a systematic
analysis for the Global Burden of Disease Study 2010'\,'. In:
\emph{The Lancet} 380.9859, pp.~2071--2094.

Wheldon, Mark C, Adrian E Raftery, Samuel J Clark, and Patrick Gerland
(2016). ``Bayesian population reconstruction of female populations for
less developed and more developed countries.'\,'. In:
\emph{Population studies} 70.1, pp.~21--37.

\hypertarget{general-migration}{%
\subsection{General Migration}\label{general-migration}}

\hypertarget{theories-of-migration}{%
\subsubsection{Theories of Migration}\label{theories-of-migration}}

Abel, Guy J. and Joel E. Cohen (2019). ``Bilateral international
migration flow estimates for 200 countries'\,'. In:
\emph{Scientific Data} 6.1, p.~82.

Brettell, Caroline and James F. Hollifield (2007).
\emph{Migration Theory: Talking across Disciplines}. New York City.

Cadena, Brian and Brian Kovak (2013).
\emph{Immigrants Equilibrate Local Labor Markets: Evidence from the Great Recession}.
Cambridge, MA: National Bureau of Economic Research.

Cooke, Thomas J, Richard Wright, and Mark Ellis (2018). ``A Prospective
on Zelinsky's Hypothesis of the Mobility Transition'\,'. In:
\emph{Geographical Review} 108.4, pp.~503--522.

Ellis, Mark, Richard Wright, and Matthew Townley (2014). ``The Great
Recession and the allure of new immigrant destinations in the United
States'\,'. In: \emph{International Migration Review} 48.1, pp.~3--33.

Hatton, Timothy and Jeffrey Williamson (2002).
\emph{What Fundamentals Drive World Migration?}. Cambridge, MA: National
Bureau of Economic Research.

King, Russell (2012). ``Geography and Migration Studies: Retrospect and
Prospect'\,'. In: \emph{Population, Space and Place} 18.2, pp.~134--153.

Massey, Douglas S, Joaquin Arango, Graeme Hugo, Ali Kouaouci, Adela
Pellegrino, et al. (1993). ``Theories of International Migration: A
Review and Appraisal'\,'. In: \emph{Population and Development Review}
19.3, p.~431.

Morris, Timothy (2017). ``Examining the influence of major life events
as drivers of residential mobility and neighbourhood transitions'\,'.
In: \emph{Demographic Research} 36, pp.~1015--1038.

Wang, Sharron, Arthur Sakamoto, Sharron Xuanren Wang, and Arthur
Sakamoto (2016). ``Did the Great Recession Downsize Immigrants and
Native-Born Americans Differently? Unemployment Differentials by
Nativity, Race and Gender from 2007 to 2013 in the U.S.'\,'. In:
\emph{Social Sciences} 5.3, p.~49.

Wimmer, Andreas and Nina Glick Schiller (2003). ``Methodological
Nationalism, the Social Sciences, and the Study of Migration: An Essay
in Historical Epistemology'\,'. In:
\emph{The International Migration Review} 37.3, pp.~576--610.

\hypertarget{assimilation-and-incorporation}{%
\subsubsection{Assimilation and
Incorporation}\label{assimilation-and-incorporation}}

Alba, R and V Nee (1997). ``Rethinking assimilation theory for a new era
of immigration.'\,'. In: \emph{The International migration review} 31.4,
pp.~826--74.

Brubaker, R. (2001). ``The return of assimilation? Changing perspectives
on immigration and its sequels in France, Germany, and the United
States'\,'. In: \emph{Ethnic and Racial Studies} 24.4, pp.~531--548.

Gans, Herbert J. (1997). ``Toward a reconciliation of ``assimilation''
and ``pluralism'': The interplay of acculturation and ethnic
retention'\,'. In: \emph{International Migration Review} 31.4,
pp.~875--892.

Gordon, Milton M. (1964).
\emph{Assimilation in American life : the role of race, religion, and national origins.}.
Oxford University Press, p.~276.

Jimenez, Tomas R. (2017).
\emph{The other side of assimilation : how immigrants are changing American life}
, p.~275.

Massey, Douglas S. and Magaly. Sanchez (2010).
\emph{Brokered boundaries : creating immigrant identity in anti-immigrant times}.
Russell Sage Foundation, p.~305.

Portes, Alejandro and Min Zhou (1993). ``The New Second Generation:
Segmented Assimilation and Its Variants'\,'. In:
\emph{The Annals of the American Academy of Political and Social Science}
530, pp.~74--96.

Rumbaut, Ruben G. (1997). ``Paradoxes (and Orthodoxies) of
Assimilation'\,'. In: \emph{Sociological Perspectives} 40.3,
pp.~483--511.

Waldinger, Roger and Peter Catron (2016). ``Modes of incorporation: a
conceptual and empirical critique'\,'. In:
\emph{Journal of Ethnic and Migration Studies} 42.1, pp.~23--53.

Waldinger, Roger and Cynthia Feliciano (2004). ``Will the new second
generation experience `downward assimilation'? Segmented assimilation
re-assessed'\,'. In: \emph{Ethnic and Racial Studies} 27.3,
pp.~376--402.

Waters, Mary C. and Marisa Gerstein Pineau (2016).
\emph{The integration of immigrants into American society}. National
Academies Press, pp.~1--458.

\hypertarget{internal-migration}{%
\subsubsection{Internal Migration}\label{internal-migration}}

Cohen, Jeffrey H. and Bernardo Ramirez Rios (2016). ``Internal Migration
in Oaxaca: Its Role and Value to Rural Movers'\,'. In:
\emph{International Journal of Sociology} 46.3, pp.~223--235.

Cooke, Thomas J. (2013). ``Internal Migration in Decline'\,'. In:
\emph{The Professional Geographer} 65.4, pp.~664--675.

Ellis, Mark (2012). ``Reinventing US Internal Migration Studies in the
Age of International Migration'\,'. In:
\emph{Population, Space and Place} 18.2, pp.~196--208.

Frey, William and Julie Park (2011). ``Migration and Dispersal of
Hispanic and Asian Groups: An Analysis of the 2006-2008 Multiyear
American Community Survey'\,'. In: \emph{SSRN Electronic Journal}.

King, Russell and Ronald Skeldon (2010). ```Mind the Gap!' Integrating
approaches to internal and international migration'\,'. In:
\emph{Journal of Ethnic and Migration Studies} 36.10, pp.~1619--1646.

Kritz, Mary M. and Douglas T. Gurak (2010). ``Foreign-Born Out-Migration
from New Destinations: The Effects of Economic Conditions and Nativity
Concentration'\,'. In: \emph{SSRN Electronic Journal}.

Molloy, Raven, Christopher L Smith, and Abigail Wozniak (2011).
``Internal Migration in the United States'\,'. In:
\emph{Journal of Economic Perspectives} 25.3, pp.~173--196.

Skeldon, Ronald (2012). ``Migration Transitions Revisited: Their
Continued Relevance for The Development of Migration Theory'\,'. In:
\emph{Population, Space and Place} 18.2, pp.~154--166.

\hypertarget{context}{%
\subsubsection{Context}\label{context}}

Clevenger, Casey, Amelia Seraphia Derr, Wendy Cadge, and Sara Curran
(2014). ``How do social service providers view recent immigrants?
Perspectives from Portland, Maine, and Olympia, Washington'\,'. In:
\emph{Journal of Immigrant \& Refugee Studies} 12.1, pp.~67--86.

Ellis, Mark and Gunnar Almgren (2009). ``Local Contexts of Immigrant and
Second-Generation Integration in the United States'\,'. In:
\emph{Journal of Ethnic and Migration Studies} 35.7, pp.~1059--1076.

Flores, Stella M. and Jorge Chapa (2009). ``Latino immigrant access to
higher education in a bipolar context of reception'\,'. In:
\emph{Journal of Hispanic Higher Education} 8.1, pp.~90--109.

Jaworsky, Bernadette Nadya, Peggy Levitt, Wendy Cadge, Jessica
Hejtmanek, and Sara R Curran (2012). ``New perspectives on immigrant
contexts of reception: The cultural armature of cities.'\,'. In:
\emph{Nordic Journal of Migration Research} 2.1, pp.~78--88.

Luthra, Renee, Thomas Soehl, and Roger Waldinger (2018).
``Reconceptualizing Context: A Multilevel Model of the Context of
Reception and Second‐Generation Educational Attainment'\,'. In:
\emph{International Migration Review} 52.3, pp.~898--928.

Menjivar, Cecilia (1997). ``Immigrant kinship networks and the impact of
the receiving context: Salvadorans in San Francisco in the early
1990s'\,'. In: \emph{Social Problems} 44.1, pp.~104--123.

Rumbaut, Ruben G. (2008). ``Reaping what you sow: Immigration, youth,
and reactive ethnicity'\,'. In: \emph{Applied Developmental Science}
12.2, pp.~108--111.

\hypertarget{hispanic-migration-in-the-united-states}{%
\subsection{Hispanic Migration in the United
States}\label{hispanic-migration-in-the-united-states}}

\hypertarget{historical-demographic}{%
\subsubsection{Historical Demographic}\label{historical-demographic}}

Caban, Pedro (2003). ``From Challenge to Absorption: The Changing Face
of Latina and Latino Studies'\,'. In: \emph{Centro Journal} 15.2.

Corwin, Arthur F. (1973a).
\emph{Mexican Emigration History, 1900-1970: Literature and Research}.

-------- (1973b). ``Mexican-American History: An Assessment'\,'. In:
\emph{Pacific Historical Review} 42.3, pp.~269--308.

Flores, Antonio (2017). \emph{Facts on Latinos in America}. Washington
D.C.: Pew Hispanic.

Gonzalez-Barrera, Ana and Mark Hugo Lopez (2013).
\emph{A Demographic Portrait of Mexican-Origin Hispanics in the United States}.
Washington D.C.: Pew Hispanic, p.~22.

Hall, Matthew, Laura Tach, and Barrett A. Lee (2016). ``Trajectories of
Ethnoracial Diversity in American Communities, 1980-2010'\,'. In:
\emph{Population and Development Review} 42.2, pp.~271--297.

Kandel, William and John Cromartie (2004).
\emph{New Patterns of Hispanic Settlement in Rural America}. Washington
DC: United States Department of Agriculture.

Lichter, Daniel T. (2012). ``Immigration and the New Racial Diversity in
Rural America'\,'. In: \emph{Rural Sociology} 77.1, pp.~3--35.

\hypertarget{policy-changes-immigration-and-undocumented-status}{%
\subsubsection{Policy Changes, Immigration, and Undocumented
Status}\label{policy-changes-immigration-and-undocumented-status}}

Castles, Stephen, Hein de Haas, and Mark J. Miller (2013).
\emph{The age of migration : international population movements in the modern world}.
5th ed. Guilford Press, p.~401.

Durand, Jorge and Douglas S Massey (2010). ``New World Orders:
Continuities and Changes in Latin American Migration.'\,'. In:
\emph{The Annals of the American Academy of Political and Social Science}
630.1, pp.~20--52.

Golash-Boza, Tanya (2009). ``A Confluence of Interests in Immigration
Enforcement: How Politicians, the Media, and Corporations Profit from
Immigration Policies Destined to Fail'\,'. In: \emph{Sociology Compass}
3.2, pp.~283--294.

Gonzales, Roberto G. (2011). ``Learning to Be Illegal'\,'. In:
\emph{American Sociological Review} 76.4, pp.~602--619.

Hainmueller, Jens, Duncan Lawrence, Linna Marten, Bernard Black, Lucila
Figueroa, et al. (2017). ``Protecting unauthorized immigrant mothers
improves their children's mental health'\,'. In: \emph{Science}
357.6355, pp.~1041--1044.

Hall, Matthew, Emily Greenman, and George Farkas (2010). ``Legal Status
and Wage Disparities for Mexican Immigrants.'\,'. In:
\emph{Social forces; a scientific medium of social study and interpretation}
89.2, pp.~491--513.

Hall, Matthew, Emily Greenman, and Youngmin Yi (2019). ``Job Mobility
among Unauthorized Immigrant Workers'\,'. In: \emph{Social Forces} 97.3,
pp.~999--1028.

Kerwin, Donald (2010).
\emph{More than IRCA: U.S. Legalization Programs and the Current Policy Debate | migrationpolicy.org}.
Washington D.C: Migration Policy Institute.

Massey, Douglas S, Nolan J. Malone, and Jorge. Durand (2002).
\emph{Beyond smoke and mirrors : Mexican immigration in an era of economic integration}.
Washington D.C.: Russell Sage Foundation, p.~199.

Massey, Douglas S, Karen A Pren, and Jorge Durand (2009). ``Nuevos
escenarios de la migración México-Estados Unidos. Las consecuencias de
la guerra antiinmigrante'\,'. In: \emph{Papeles de poblacion} 15.61,
pp.~101--128.

Menjivar, Cecilia (2006).
\emph{Liminal legality: Salvadoran and Guatemalan immigrants' lives in the United States}.

Van Hook, Jennifer, James D. Bachmeier, Donna L. Coffman, and Ofer Harel
(2015). ``Can We Spin Straw Into Gold? An Evaluation of Immigrant Legal
Status Imputation Approaches'\,'. In: \emph{Demography} 52.1,
pp.~329--354.

\hypertarget{accumuolation-and-networks}{%
\subsubsection{Accumuolation and
Networks}\label{accumuolation-and-networks}}

Alarcon, Rafael, Luis Escala-Rabadan, Olga Odgers, Dick Cluster, and
Roger David Waldinger (2016).
\emph{Making Los Angeles home : the integration of Mexican immigrants in the United States}
, p.~259.

Anguiano-Tellez, Maria Eugenia and Alma Paola Trejo-Pena (2007).
``Vigilante and control at the U.S.-Mexico border region: The new routes
of international flows.'\,'. In: \emph{Papeles de Poblacion} 13.51,
pp.~45--75.

Crowley, Martha and Daniel T. Lichter (2010). ``Social Disorganization
in New Latino Destinations?'\,'. In: \emph{Rural Sociology} 74.4,
pp.~573--604.

Durand, Jorge (1986). ``Circuitos migratorios en el occidente de
Mexico'\,'. In: \emph{Revue européenne des migrations internationales}
2.2, pp.~49--67.

Garip, Filiz (2016).
\emph{On the move : changing mechanisms of Mexico-U.S. migration}.
Princeton, NJ: Princeton University Press, p.~294.

Lee, Barrett A, Michael J.R. Martin, and Matthew Hall (2017).
``Solamente Mexicanos? Patterns and sources of Hispanic diversity in
U.S. metropolitan areas'\,'. In: \emph{Social Science Research} 68,
pp.~117--131.

Levitt, Peggy and Nina Glick Schiller (2006). ``Conceptualizing
Simultaneity: A Transnational Social Field Perspective on Society'\,'.
In: \emph{International Migration Review} 38.3, pp.~1002--1039.

Massey, Douglas S. (1986). ``The Settlement Process Among Mexican
Migrants to the United States'\,'. In:
\emph{American Sociological Review} 51.5, p.~670.

Riosmena, Fernando, Rebeca Wong, and Alberto Palloni (2013). ``Migration
Selection, Protection, and Acculturation in Health: A Binational
Perspective on Older Adults'\,'. In: \emph{Demography} 50.3,
pp.~1039--1064.

\hypertarget{new-destinations}{%
\subsubsection{New Destinations}\label{new-destinations}}

Barcus, Holly R. (2007). ``The Emergence of New Hispanic Settlement
Patterns in Appalachia'\,'. In: \emph{The Professional Geographer} 59.3,
pp.~298--315.

Crowley, Martha, Daniel T. Lichter, and Zhenchao Qian (2006). ``Beyond
Gateway Cities: Economic Restructuring and Poverty Among Mexican
Immigrant Families and Children'\,'. In: \emph{Family Relations} 55.3,
pp.~345--360.

Durand, Jorge, Douglas S. Massey, and Emilio A. Parrado (1999). ``The
New Era of Mexican Migration to the United States'\,'. In:
\emph{The Journal of American History} 86.2, p.~518.

Durand, Jorge, Douglas Massey, and Fernando Charvet (2000). ``The
Changing Geography of Mexican Immigration to the US:1910-1996'\,'. In:
\emph{Social Science Quarterly} 81.1, pp.~1--15.

Foster, Thomas, Mark Ellis, and Lee Fiorio (2018). ``Foreign-Born and
Native-Born Migration in the U.S.: Evidence from\ldots'\,'. In:
\emph{Census Working Papers}.

Hall, Matthew and Jacob Hibel (2017). ``Latino Students and White
Migration from School Districts, 1980-2010'\,'. In:
\emph{Social Problems} 64.4, pp.~457--475.

Hernandez-Leon, Ruben and Victor Zuniga (2000). ``Making carpet by the
mile: The emergence of a Mexican immigrant community in an industrial
region of the U.S. historic South'\,'. In:
\emph{Social Science Quarterly} 81.1, pp.~49--66.

Kandel, William and Emilio A. Parrado (2005). ``Restructuring of the US
Meat Processing Industry and New Hispanic Migrant Destinations'\,'. In:
\emph{Population and Development Review} 31.3, pp.~447--471.

Lichter, Daniel T. and Kenneth M. Johnson (2009). ``Immigrant Gateways
and Hispanic Migration to New Destinations'\,'. In:
\emph{International Migration Review} 43.3, pp.~496--518.

Light, Ivan and Michael Francis Johnston (2009). ``The metropolitan
dispersion of Mexican immigrants in the United States, 1980 to 2000'\,'.
In: \emph{Journal of Ethnic and Migration Studies} 35.1, pp.~3--18.

Massey, Douglas S. (2008).
\emph{New faces in new places : the changing geography of American immigration}.
Russell Sage Foundation, p.~370.

Massey, Douglas S, Jacob S. Rugh, and Karen A. Pren (2010). ``The
Geography of Undocumented Mexican Migration'\,'. In:
\emph{Mexican Studies/Estudios Mexicanos} 26.1, pp.~129--152.

Parrado, Emilio A. and Chenoa A. Flippen (2016). ``The Departed:
Deportations and Out-Migration among Latino Immigrants in North Carolina
after the Great Recession'\,'. In:
\emph{The ANNALS of the American Academy of Political and Social Science}
666.1. Ed. by Katharine M. Donato and Douglas S. Massey, pp.~131--147.

Singer, Audrey (2004). \emph{The Rise of New Immigrant Gateways}.
Washington DC: Brookings.

-------- (2015). \emph{Metropolitan immigrant gateways revisited, 2014}.
Washington DC: Brookings.




\end{document}

\makeatletter
\def\@maketitle{%
  \newpage
%  \null
%  \vskip 2em%
%  \begin{center}%
  \let \footnote \thanks
    {\fontsize{18}{20}\selectfont\raggedright  \setlength{\parindent}{0pt} \@title \par}%
}
%\fi
\makeatother
