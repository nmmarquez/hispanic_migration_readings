\documentclass[11pt,]{article}
\usepackage[margin=1in]{geometry}
\newcommand*{\authorfont}{\fontfamily{phv}\selectfont}
\usepackage[]{mathpazo}
\usepackage{abstract}
\renewcommand{\abstractname}{}    % clear the title
\renewcommand{\absnamepos}{empty} % originally center
\newcommand{\blankline}{\quad\pagebreak[2]}

\providecommand{\tightlist}{%
  \setlength{\itemsep}{0pt}\setlength{\parskip}{0pt}} 
\usepackage{longtable,booktabs}

\usepackage{parskip}
\usepackage{titlesec}
\titlespacing\section{0pt}{12pt plus 4pt minus 2pt}{6pt plus 2pt minus 2pt}
\titlespacing\subsection{0pt}{12pt plus 4pt minus 2pt}{6pt plus 2pt minus 2pt}

\titleformat*{\subsubsection}{\normalsize\itshape}

\usepackage{titling}
\setlength{\droptitle}{-.25cm}

%\setlength{\parindent}{0pt}
%\setlength{\parskip}{6pt plus 2pt minus 1pt}
%\setlength{\emergencystretch}{3em}  % prevent overfull lines 

\usepackage[T1]{fontenc}
\usepackage[utf8]{inputenc}

\usepackage{fancyhdr}
\pagestyle{fancy}
\usepackage{lastpage}
\renewcommand{\headrulewidth}{0.3pt}
\renewcommand{\footrulewidth}{0.0pt} 
\lhead{}
\chead{}
\rhead{\footnotesize SOC 800: Geographic Redistiribution of Latin
Americans in the United States -- Fall 2019}
\lfoot{}
\cfoot{\small \thepage/\pageref*{LastPage}}
\rfoot{}

\fancypagestyle{firststyle}
{
\renewcommand{\headrulewidth}{0pt}%
   \fancyhf{}
   \fancyfoot[C]{\small \thepage/\pageref*{LastPage}}
}

%\def\labelitemi{--}
%\usepackage{enumitem}
%\setitemize[0]{leftmargin=25pt}
%\setenumerate[0]{leftmargin=25pt}




\makeatletter
\@ifpackageloaded{hyperref}{}{%
\ifxetex
  \usepackage[setpagesize=false, % page size defined by xetex
              unicode=false, % unicode breaks when used with xetex
              xetex]{hyperref}
\else
  \usepackage[unicode=true]{hyperref}
\fi
}
\@ifpackageloaded{color}{
    \PassOptionsToPackage{usenames,dvipsnames}{color}
}{%
    \usepackage[usenames,dvipsnames]{color}
}
\makeatother
\hypersetup{breaklinks=true,
            bookmarks=true,
            pdfauthor={ ()},
             pdfkeywords = {},  
            pdftitle={SOC 800: Geographic Redistiribution of Latin
Americans in the United States},
            colorlinks=true,
            citecolor=blue,
            urlcolor=blue,
            linkcolor=magenta,
            pdfborder={0 0 0}}
\urlstyle{same}  % don't use monospace font for urls


\setcounter{secnumdepth}{0}





\usepackage{setspace}

\title{SOC 800: Geographic Redistiribution of Latin Americans in the
United States}
\author{Neal Marquez}
\date{Fall 2019}


\begin{document}  

		\maketitle
		
	
		\thispagestyle{firststyle}

%	\thispagestyle{empty}


	\noindent \begin{tabular*}{\textwidth}{ @{\extracolsep{\fill}} lr @{\extracolsep{\fill}}}


E-mail: \texttt{\href{mailto:nmarquez@u.washington.edu}{\nolinkurl{nmarquez@u.washington.edu}}} & Web: TBD\\
Office Hours: TBD  &  Class Hours: TBD\\
Office: TBD  & Class Room: TBD\\
	&  \\
	\hline
	\end{tabular*}
	
\vspace{2mm}
	


\hypertarget{course-description}{%
\section{Course Description}\label{course-description}}

Historically Latin America, and particularly Mexico, has been a large
source of the overall immigration that has occurred in the United
States. The relatively young population of Latin American countries and
the demand for low wage labor in the United States have acted as push
and pull factors for migration respectively. Geographic distribution,
however, has widely varied for immigration events. In addition to
traditional immigrant hubs Latin American immigrants into the United
States have founded large populations On the Southern Border region and
Chicago, in the first wave of immigration events, and at the turn of
century more towards South and South Eastern States.

Most recently, migration from Mexico, which has the largest single
foreign born population in the United States, has come to a tapering
off, while migration from other Latin American countries, such as
Central American countries, have begun to pick up. Despite this,
populations of foreign born Mexicans continue to migrate within the
United States shifting the locations where traditional migrant hubs were
found. The continued migration of the Mexican foreig born population has
not well been theorized upon and, as this population ages, we would
expect the push and pull factors for migration events to evolve from
predominantly labor concerns, to potentially housing, affordability, and
health concerns.

In addition, how changes in the foreign born Mexican population affect
other racial and ethinic groups in recent years is less well studied.
While traditionally, native Hispanic populations have grown alongside
foreign born populations, more recent research shows these populations
moving in different directions. In this Syllabus, we go over several of
the historical factors that have led to large stocks of Latin Americans
in the United States, how the historical pull factors may be shifting,
and how this affects other racial/ethnic group dynamics such as
segregation and migration.

\hypertarget{course-objectives}{%
\section{Course Objectives}\label{course-objectives}}

\begin{enumerate}
\def\labelenumi{\arabic{enumi}.}
\item
  Read Through Breif History of Latin American Migration in the United
  States
\item
  Evaluate current litearture on studies involving new destinations and
  new realtionships between ethnic/racial groups.
\item
  Begin to bring in data sources that fill the gap on the current
  discussion.
\end{enumerate}

\hypertarget{demography}{%
\section{Demography}\label{demography}}

\hypertarget{overview-and-evolution}{%
\subsection{Overview and Evolution}\label{overview-and-evolution}}

Caldwell, J.C. (1996). ``Demography and Social Science'\,'. In:
\emph{Population Studies} 50.3, pp.~305--333.

Coale, Ansley J. (1975). ``The demographic transition'\,'. In:
\emph{The Population Debate: Dimensions and Perspectives} 1, pp.~53--72.

Greenhalgh, Susan (2009). ``The Social Construction of Population
Science: An Intellectual, Institutional, and Political History of
Twentieth-Century Demography'\,'. In:
\emph{Comparative Studies in Society and History} 38.01, p.~26.

Greg J. Duncan, Greg j. (2008). ``When to Promote, and When to Avoid, a
Population Perspective'\,'. In: \emph{Demography} 45.4, pp.~763--784.

Lee, Ronald (2003). ``The Demographic Transition: Three Centuries of
Fundamental Change'\,'. In: \emph{Journal of Economic Perspectives}
17.4, pp.~167--190.

Moffitt, Robert (2005). ``Remarks on the analysis of causal
relationships in population research.'\,'. In: \emph{Demography} 42.1,
pp.~91--108.

Preston, Samuel H. (1993). ``The Contours of Demography: Estimates and
Projections'\,'. In: \emph{Demography} 30.4, p.~593.

Tienda, Marta (2002). ``Demography and the Social Contract'\,'. In:
\emph{Demography} 39.4, pp.~587--616.

Xie, Yu (2000). ``Demography: Past, Present, and Future'\,'. In:
\emph{Journal of the American Statistical Association} 95.450, p.~670.

\hypertarget{mortaility}{%
\subsection{Mortaility}\label{mortaility}}

Caldwell, John C. (1986). ``Routes to Low Mortality in Poor
Countries'\,'. In: \emph{Population and Development Review} 12.2,
p.~171.

-------- (1990). ``Cultural and Social Factors Influencing Mortality
Levels in Developing Countries'\,'. In:
\emph{The ANNALS of the American Academy of Political and Social Science}
510.1, pp.~44--59.

Clark, Shelley and Dana Hamplová (2013). ``Single Motherhood and Child
Mortality in Sub-Saharan Africa: A Life Course Perspective'\,'. In:
\emph{Demography} 50.5, pp.~1521--1549.

Colgrove, James (2002). ``The McKeown thesis: a historical controversy
and its enduring influence.'\,'. In:
\emph{American journal of public health} 92.5, pp.~725--9.

Kuhn, Randall (2010). ``Routes to low mortality in poor countries
revisited.'\,'. In: \emph{Population and development review} 36.4,
pp.~655--92.

Link, Bruce G and Jo C Phelan (2002). ``McKeown and the idea that social
conditions are fundamental causes of disease.'\,'. In:
\emph{American journal of public health} 92.5, pp.~730--2.

Markides, Kyriakos S and Karl Eschbach (2005). ``Aging, migration, and
mortality: current status of research on the Hispanic paradox.'\,'. In:
\emph{The journals of gerontology. Series B, Psychological sciences and social sciences}
60 Spec No, pp.~68--75.

McKeown, Thomas, R. G. Brown, and R. G. Record (1972). ``An
Interpretation of the Modern Rise of Population in Europe'\,'. In:
\emph{Population Studies} 26.3, p.~345.

Olshansky, S J and A B Ault (1986). ``The fourth stage of the
epidemiologic transition: the age of delayed degenerative diseases.'\,'.
In: \emph{The Milbank quarterly} 64.3, pp.~355--91.

Omran, Abdel R (2005). ``The epidemiologic transition: a theory of the
epidemiology of population change. 1971.'\,'. In:
\emph{The Milbank quarterly} 83.4, pp.~731--57.

Preston, Samuel H. (1996). ``Population Studies of Mortality'\,'. In:
\emph{Population Studies} 50.3, pp.~525--536.

Szreter, Simon (2002). ``Rethinking McKeown: the relationship between
public health and social change.'\,'. In:
\emph{American journal of public health} 92.5, pp.~722--5.

\hypertarget{child-mortaility}{%
\subsection{Child Mortaility}\label{child-mortaility}}

Finch, Brian Karl (2003). ``Early Origins of the Gradient: The
Relationship Between Socioeconomic Status and Infant Mortality in the
United States'\,'. In: \emph{Demography} 40.4, pp.~675--699.

Geronimus, A T (1992). ``The weathering hypothesis and the health of
African-American women and infants: evidence and speculations.'\,'. In:
\emph{Ethnicity \& disease} 2.3, pp.~207--21.

Gortmaker, Steven L. and Paul H. Wise (1997). ``The First Injustice:
Socioeconomic Disparities, Health Services Technology, and Infant
Mortality'\,'. In: \emph{Annual Review of Sociology} 23.1, pp.~147--170.

Hamilton, Erin R, Andrés Villarreal, and Robert A. Hummer (2009).
``Mother's, Household, and Community U.S. Migration Experience and
Infant Mortality in Rural and Urban Mexico'\,'. In:
\emph{Population Research and Policy Review} 28.2, pp.~123--142.

Hummer, Robert A, Daniel A Powers, Starling G Pullum, Ginger L Gossman,
and W Parker Frisbie (2007). ``Paradox found (again): infant mortality
among the Mexican-origin population in the United States.'\,'. In:
\emph{Demography} 44.3, pp.~441--57.

Landale, Nancy S, R. S. Oropesa, and Bridget K. Gorman (2000).
``Migration and Infant Death: Assimilation or Selective Migration among
Puerto Ricans?'\,'. In: \emph{American Sociological Review} 65.6,
p.~888.

Palloni, Alberto and Hantamala Rafalimanana (1999). ``The Effects of
Infant Mortality on Fertility Revisited: New Evidence from Latin
America'\,'. In: \emph{Demography} 36.1, p.~41.

\hypertarget{fertility}{%
\subsection{Fertility}\label{fertility}}

Bongaarts, John (1978). ``A Framework for Analyzing the Proximate
Determinants of Fertility'\,'. In:
\emph{Population and Development Review} 4.1, p.~105.

Bongaarts, John and John Casterline (2013). ``Fertility Transition: Is
sub-Saharan Africa Different?'\,'. In:
\emph{Population and development review} 38.Suppl 1, p.~153.

Bryant, John (2007). ``Theories of Fertility Decline and the Evidence
from Development Indicators'\,'. In:
\emph{Population and Development Review} 33.1, pp.~101--127.

Easterlin, R A (1975). ``An economic framework for fertility
analysis.'\,'. In: \emph{Studies in family planning} 6.3, pp.~54--63.

Hirschman, Charles (1994). ``Why Fertility Changes'\,'. In:
\emph{Annual Review of Sociology} 20.1, pp.~203--233.

Myrskylä, Mikko, Hans-Peter Kohler, and Francesco C. Billari (2009).
``Advances in development reverse fertility declines'\,'. In:
\emph{Nature} 460.7256, pp.~741--743.

\hypertarget{social-determinants-and-inequality}{%
\subsection{Social Determinants and
Inequality}\label{social-determinants-and-inequality}}

Chetty, Raj, Nathaniel Hendren, Maggie Jones, and Sonya Porter (2018).
\emph{Race and Economic Opportunity in the United States: An Intergenerational Perspective}.
Cambridge, MA: National Bureau of Economic Research.

Chetty, Raj, Nathaniel Hendren, and Lawrence F. Katz (2016). ``The
Effects of Exposure to Better Neighborhoods on Children: New Evidence
from the Moving to Opportunity Experiment'\,'. In:
\emph{American Economic Review} 106.4, pp.~855--902.

Hayward, Mark D. and Bridget K. Gorman (2004). ``The Long Arm of
Childhood: The Influence of Early-Life Social Conditions on Men's
Mortality'\,'. In: \emph{Demography} 41.1, pp.~87--107.

Link, B. G. and J. Phelan (1995).
\emph{Social conditions as fundamental causes of disease.}.

Preston, Samuel H. (1975). ``The changing relation between mortality and
level of economic development. 1975.'\,'. In: \emph{Population Studies}
29.2, pp.~231--248.

Preston, Samuel H, Mark E. Hill, and Greg L. Drevenstedt (1998).
``Childhood conditions that predict survival to advanced ages among
African--Americans'\,'. In: \emph{Social Science \& Medicine} 47.9,
pp.~1231--1246.

Read, Jen'nan Ghazal and Bridget K. Gorman (2006). ``Gender inequalities
in US adult health: The interplay of race and ethnicity'\,'. In:
\emph{Social Science \& Medicine} 62.5, pp.~1045--1065.

Rogers, Richard G, Bethany G Everett, Jarron M Saint Onge, and Patrick M
Krueger (2010). ``Social, behavioral, and biological factors, and sex
differences in mortality.'\,'. In: \emph{Demography} 47.3, pp.~555--78.

Sampson, Robert J and Patrick Sharkey (2008). ``Neighborhood selection
and the social reproduction of concentrated racial inequality.'\,'. In:
\emph{Demography} 45.1, pp.~1--29.

Singh-Manoux, Archana and Michael Marmot (2005). ``Role of socialization
in explaining social inequalities in health'\,'. In:
\emph{Social Science \& Medicine} 60.9, pp.~2129--2133.

Szreter, Simon (1988). ``The Importance of Social Intervention in
Britain's Mortality Decline c .1850--1914: a Re-interpretation of the
Role of Public Health'\,'. In: \emph{Social History of Medicine} 1.1,
pp.~1--38.

\hypertarget{methods-and-data}{%
\subsection{Methods and Data}\label{methods-and-data}}

Bhroilchain, Maire Ni and Tim Dyson (2007). ``On Causation in
Demography: Issues and Illustrations'\,'. In:
\emph{Population and Development Review} 33.1, pp.~1--36.

Caldwell, John C. (1985).
\emph{Strengths and limitations of the survey approach for measuring and understanding fertility change: alternative possibilities.}.

Hobcraft, J, J Menken, and S Preston (1982). ``Age, period, and cohort
effects in demography: a review.'\,'. In: \emph{Population index} 48.1,
pp.~4--43.

Maine, Deborah, Lynn Freedman, Farida Shaheed, and Schuyler Frautschi
(1994). ``Risk, reproduction, and rights: The uses of reproductive
health data'\,'. In: \emph{Reproductive Health Matters} 3.6, pp.~40--51.

Palmore, James A. and Robert W. Gardner (1994).
\emph{Measuring mortality, fertility, and natural increase : a self-teaching guide to elementary measures}.
East-West Center, p.~169.

Preston, Sam, Patrick Heuveline, and Michel Guillot (2000).
\emph{Demography: Measuring and Modeling Population Processes}.
Wiley-Blackwell.

Raftery, Adrian E, Leontine Alkema, and Patrick Gerland (2014).
``Bayesian Population Projections for the United Nations.'\,'. In:
\emph{Statistical science : a review journal of the Institute of Mathematical Statistics}
29.1, pp.~58--68.

Snipp, C. Matthew (2003). ``Racial Measurement in the American Census:
Past Practices and Implications for the Future'\,'. In:
\emph{Annual Review of Sociology} 29.1, pp.~563--588.

Wachter, Kenneth W. (2014). \emph{Essential demographic methods}. 1st
ed. Harvard University Press, p.~288.

Wakefield, Jon, Geir-Arne Fuglstad, Andrea Riebler, Jessica Godwin,
Katie Wilson, et al. (2018). ``Estimating under-five mortality in space
and time in a developing world context'\,'. In:
\emph{Statistical Methods in Medical Research}, p.~096228021876798.

Wang, Haidong, Laura Dwyer-Lindgren, Katherine T Lofgren, Julie Knoll
Rajaratnam, Jacob R Marcus, et al. (2012). ``Age-specific and
sex-specific mortality in 187 countries, 1970--2010: a systematic
analysis for the Global Burden of Disease Study 2010'\,'. In:
\emph{The Lancet} 380.9859, pp.~2071--2094.

Wheldon, Mark C, Adrian E Raftery, Samuel J Clark, and Patrick Gerland
(2016). ``Bayesian population reconstruction of female populations for
less developed and more developed countries.'\,'. In:
\emph{Population studies} 70.1, pp.~21--37.

\hypertarget{general-migration}{%
\section{General Migration}\label{general-migration}}

\hypertarget{theories-of-migration}{%
\subsection{Theories of Migration}\label{theories-of-migration}}

Brettell, Caroline and James F. Hollifield (2007).
\emph{Migration Theory: Talking across Disciplines}. New York City.

Cadena, Brian and Brian Kovak (2013).
\emph{Immigrants Equilibrate Local Labor Markets: Evidence from the Great Recession}.
Cambridge, MA: National Bureau of Economic Research.

Calnan, Ray and Gary Painter (2017). ``The response of Latino immigrants
to the Great Recession: Occupational and residential (im)mobility'\,'.
In: \emph{Urban Studies} 54.11, pp.~2561--2591.

Cooke, Thomas J, Richard Wright, and Mark Ellis (2018). ``A Prospective
on Zelinsky's Hypothesis of the Mobility Transition'\,'. In:
\emph{Geographical Review} 108.4, pp.~503--522.

Ellis, Mark, Richard Wright, and Matthew Townley (2014). ``The Great
Recession and the Allure of New Immigrant Destinations in the United
States'\,'. In: \emph{International Migration Review} 48.1, pp.~3--33.

Hatton, Timothy and Jeffrey Williamson (2002).
\emph{What Fundamentals Drive World Migration?}. Cambridge, MA: National
Bureau of Economic Research.

King, Russell (2012). ``Geography and Migration Studies: Retrospect and
Prospect'\,'. In: \emph{Population, Space and Place} 18.2, pp.~134--153.

Massey, Douglas S, Joaquin Arango, Graeme Hugo, Ali Kouaouci, Adela
Pellegrino, et al. (1993). ``Theories of International Migration: A
Review and Appraisal'\,'. In: \emph{Population and Development Review}
19.3, p.~431.

Morris, Timothy (2017). ``Examining the influence of major life events
as drivers of residential mobility and neighbourhood transitions'\,'.
In: \emph{Demographic Research} 36, pp.~1015--1038.

Wang, Sharron, Arthur Sakamoto, Sharron Xuanren Wang, and Arthur
Sakamoto (2016). ``Did the Great Recession Downsize Immigrants and
Native-Born Americans Differently? Unemployment Differentials by
Nativity, Race and Gender from 2007 to 2013 in the U.S.'\,'. In:
\emph{Social Sciences} 5.3, p.~49.

Wimmer, Andreas and Nina Glick Schiller (2003). ``Methodological
Nationalism, the Social Sciences, and the Study of Migration: An Essay
in Historical Epistemology'\,'. In:
\emph{The International Migration Review} 37.3, pp.~576--610.




\end{document}

\makeatletter
\def\@maketitle{%
  \newpage
%  \null
%  \vskip 2em%
%  \begin{center}%
  \let \footnote \thanks
    {\fontsize{18}{20}\selectfont\raggedright  \setlength{\parindent}{0pt} \@title \par}%
}
%\fi
\makeatother
