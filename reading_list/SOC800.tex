\documentclass[11pt,]{article}
\usepackage[margin=1in]{geometry}
\newcommand*{\authorfont}{\fontfamily{phv}\selectfont}
\usepackage[]{mathpazo}
\usepackage{abstract}
\renewcommand{\abstractname}{}    % clear the title
\renewcommand{\absnamepos}{empty} % originally center
\newcommand{\blankline}{\quad\pagebreak[2]}

\providecommand{\tightlist}{%
  \setlength{\itemsep}{0pt}\setlength{\parskip}{0pt}} 
\usepackage{longtable,booktabs}

\usepackage{parskip}
\usepackage{titlesec}
\titlespacing\section{0pt}{12pt plus 4pt minus 2pt}{6pt plus 2pt minus 2pt}
\titlespacing\subsection{0pt}{12pt plus 4pt minus 2pt}{6pt plus 2pt minus 2pt}

\titleformat*{\subsubsection}{\normalsize\itshape}

\usepackage{titling}
\setlength{\droptitle}{-.25cm}

%\setlength{\parindent}{0pt}
%\setlength{\parskip}{6pt plus 2pt minus 1pt}
%\setlength{\emergencystretch}{3em}  % prevent overfull lines 

\usepackage[T1]{fontenc}
\usepackage[utf8]{inputenc}

\usepackage{fancyhdr}
\pagestyle{fancy}
\usepackage{lastpage}
\renewcommand{\headrulewidth}{0.3pt}
\renewcommand{\footrulewidth}{0.0pt} 
\lhead{}
\chead{}
\rhead{\footnotesize SOC 800: Geographic Redistiribution of Latin Americans in the United
States -- Fall 2019}
\lfoot{}
\cfoot{\small \thepage/\pageref*{LastPage}}
\rfoot{}

\fancypagestyle{firststyle}
{
\renewcommand{\headrulewidth}{0pt}%
   \fancyhf{}
   \fancyfoot[C]{\small \thepage/\pageref*{LastPage}}
}

%\def\labelitemi{--}
%\usepackage{enumitem}
%\setitemize[0]{leftmargin=25pt}
%\setenumerate[0]{leftmargin=25pt}




\makeatletter
\@ifpackageloaded{hyperref}{}{%
\ifxetex
  \usepackage[setpagesize=false, % page size defined by xetex
              unicode=false, % unicode breaks when used with xetex
              xetex]{hyperref}
\else
  \usepackage[unicode=true]{hyperref}
\fi
}
\@ifpackageloaded{color}{
    \PassOptionsToPackage{usenames,dvipsnames}{color}
}{%
    \usepackage[usenames,dvipsnames]{color}
}
\makeatother
\hypersetup{breaklinks=true,
            bookmarks=true,
            pdfauthor={ ()},
             pdfkeywords = {},  
            pdftitle={SOC 800: Geographic Redistiribution of Latin Americans in the United
States},
            colorlinks=true,
            citecolor=blue,
            urlcolor=blue,
            linkcolor=magenta,
            pdfborder={0 0 0}}
\urlstyle{same}  % don't use monospace font for urls


\setcounter{secnumdepth}{0}





\usepackage{setspace}

\title{SOC 800: Geographic Redistiribution of Latin Americans in the United
States}
\author{Neal Marquez}
\date{Fall 2019}


\begin{document}  

		\maketitle
		
	
		\thispagestyle{firststyle}

%	\thispagestyle{empty}


	\noindent \begin{tabular*}{\textwidth}{ @{\extracolsep{\fill}} lr @{\extracolsep{\fill}}}


E-mail: \texttt{\href{mailto:nmarquez@u.washington.edu}{\nolinkurl{nmarquez@u.washington.edu}}} & Web: TBD\\
Office Hours: TBD  &  Class Hours: TBD\\
Office: TBD  & Class Room: TBD\\
	&  \\
	\hline
	\end{tabular*}
	
\vspace{2mm}
	


\hypertarget{course-description}{%
\section{Course Description}\label{course-description}}

Historically Latin America, and particularly Mexico, has been a large
source of the overall immigration that has occurred in the United
States. The relatively young population of Latin American countries and
the demand for low wage labor in the United States have acted as push
and pull factors for migration respectively. Geographic distribution,
however, has widely varied for immigration events. In addition to
traditional immigrant hubs Latin American immigrants into the United
States have founded large populations On the Southern Border region and
Chicago, in the first wave of immigration events, and at the turn of
century more towards South and South Eastern States.

Most recently, migration from Mexico, which has the largest single
foreign born population in the United States, has come to a tapering
off, while migration from other Latin American countries, such as
Central American countries, have begun to pick up. Despite this,
populations of foreign born Mexicans continue to migrate within the
United States shifting the locations where traditional migrant hubs were
found. The continued migration of the Mexican foreig born population has
not well been theorized upon and, as this population ages, we would
expect the push and pull factors for migration events to evolve from
predominantly labor concerns, to potentially housing, affordability, and
health concerns.

In addition, how changes in the foreign born Mexican population affect
other racial and ethinic groups in recent years is less well studied.
While traditionally, native Hispanic populations have grown alongside
foreign born populations, more recent research shows these populations
moving in different directions. In this Syllabus, we go over several of
the historical factors that have led to large stocks of Latin Americans
in the United States, how the historical pull factors may be shifting,
and how this affects other racial/ethnic group dynamics such as
segregation and migration.

\hypertarget{course-objectives}{%
\section{Course Objectives}\label{course-objectives}}

\begin{enumerate}
\def\labelenumi{\arabic{enumi}.}
\item
  Read Through Breif History of Latin American Migration in the United
  States
\item
  Evaluate current litearture on studies involving new destinations and
  new realtionships between ethnic/racial groups.
\item
  Begin to bring in data sources that fill the gap on the current
  discussion.
\end{enumerate}

\hypertarget{reading-schedule}{%
\section{Reading Schedule}\label{reading-schedule}}

\hypertarget{week-1-historical-migration}{%
\subsection{Week 1: Historical
Migration}\label{week-1-historical-migration}}

Corwin, Arthur F. (1973). ``Mexican-American History: An Assessment''.
In: \emph{Pacific Historical Review} 42.3, pp.~269--308. ISSN: 00308684.
DOI: 10.2307/3637678.
\url{http://phr.ucpress.edu/cgi/doi/10.2307/3637678}.

--------
\emph{Mexican Emigration History, 1900-1970: Literature and Research}.
DOI: 10.2307/2502707. \url{https://www.jstor.org/stable/2502707}.

Durand, Jorge and Douglas S Massey (2010). ``New World Orders:
Continuities and Changes in Latin American Migration.''. In:
\emph{The Annals of the American Academy of Political and Social Science}
630.1, pp.~20--52. ISSN: 0002-7162. DOI: 10.1177/0002716210368102.
\url{http://www.ncbi.nlm.nih.gov/pubmed/20814591 http://www.pubmedcentral.nih.gov/articlerender.fcgi?artid=PMC2931359}.

Kerwin, Donald (2010).
\emph{More than IRCA: U.S. Legalization Programs and the Current Policy Debate | migrationpolicy.org}.
Washington D.C: Migration Policy Institute.
\url{https://www.migrationpolicy.org/research/us-legalization-programs-by-the-numbers}.

\hypertarget{week-2-new-destinations}{%
\subsection{Week 2: New Destinations}\label{week-2-new-destinations}}

Barcus, Holly R. (2007). ``The Emergence of New Hispanic Settlement
Patterns in Appalachia''. In: \emph{The Professional Geographer} 59.3,
pp.~298--315. ISSN: 0033-0124. DOI: 10.1111/j.1467-9272.2007.00614.x.
\url{http://www.tandfonline.com/doi/abs/10.1111/j.1467-9272.2007.00614.x}.

Durand, Jorge, Douglas S. Massey, and Emilio A. Parrado (1999). ``The
New Era of Mexican Migration to the United States''. In:
\emph{The Journal of American History} 86.2, p.~518. ISSN: 00218723.
DOI: 10.2307/2567043.
\url{https://academic.oup.com/jah/article-lookup/doi/10.2307/2567043}.

Massey, Douglas S, Jacob S. Rugh, and Karen A. Pren (2010). ``The
Geography of Undocumented Mexican Migration''. In:
\emph{Mexican Studies/Estudios Mexicanos} 26.1, pp.~129--152. ISSN:
0742-9797. DOI: 10.1525/msem.2010.26.1.129.
\url{http://msem.ucpress.edu/cgi/doi/10.1525/msem.2010.26.1.129}.

Singer, Audrey (2004). \emph{The Rise of New Immigrant Gateways}.
Washington DC: Brookings.
\url{https://www.brookings.edu/research/the-rise-of-new-immigrant-gateways/}.

\hypertarget{week-3-mexican-american-demography}{%
\subsection{Week 3: Mexican-American
Demography}\label{week-3-mexican-american-demography}}

Gonzalez-Barrera, Ana and Mark Hugo Lopez (2013).
\emph{A Demographic Portrait of Mexican-Origin Hispanics in the United States}.
Washington D.C.: Pew Hispanic, p.~22.

Jiménez, Tomás R (2008). ``Mexican immigrant replenishment and the
continuing significance of ethnicity and race.''. In:
\emph{AJS; American journal of sociology} 113.6, pp.~1527--67. ISSN:
0002-9602. \url{http://www.ncbi.nlm.nih.gov/pubmed/19044142}.

Johnson, Kenneth M. and Daniel T. Lichter (2008). ``Natural Increase: A
New Source of Population Growth in Emerging Hispanic Destinations in the
United States''. In: \emph{Population and Development Review} 34.2,
pp.~327--346. ISSN: 0098-7921. DOI: 10.1111/j.1728-4457.2008.00222.x.
\url{http://doi.wiley.com/10.1111/j.1728-4457.2008.00222.x}.

-------- (2016). ``Diverging Demography: Hispanic and Non-Hispanic
Contributions to U.S. Population Redistribution and Diversity''. In:
\emph{Population Research and Policy Review} 35.5, pp.~705--725. ISSN:
0167-5923. DOI: 10.1007/s11113-016-9403-3.
\url{http://link.springer.com/10.1007/s11113-016-9403-3}.

Krogstad, Jens Manuel and Mark Hugo Lopez (2014).
\emph{Hispanic Nativity Shift}. Washington D.C.: Pew Hispanic.
\url{https://www.pewhispanic.org/2014/04/29/hispanic-nativity-shift/}.

Lichter, Daniel T, Kenneth M. Johnson, Richard N. Turner, and Allison
Churilla (2012). ``Hispanic Assimilation and Fertility in New U.S.
Destinations''. In: \emph{International Migration Review} 46.4,
pp.~767--791. DOI: 10.1111/imre.12000.
\url{http://journals.sagepub.com/doi/10.1111/imre.12000}.

\hypertarget{week-4-health-and-migration}{%
\subsection{Week 4: Health and
Migration}\label{week-4-health-and-migration}}

Beltrán-Sánchez, Hiram, Alberto Palloni, Fernando Riosmena, and Rebeca
Wong (2016). ``SES Gradients Among Mexicans in the United States and in
Mexico: A New Twist to the Hispanic Paradox?''. In: \emph{Demography}
53.5, pp.~1555--1581. ISSN: 0070-3370. DOI: 10.1007/s13524-016-0508-4.
\url{http://link.springer.com/10.1007/s13524-016-0508-4}.

Crimmins, Eileen M. and Yuan S. Zhang (2019). ``Aging Populations,
Mortality, and Life Expectancy''. In: \emph{Annual Review of Sociology}
45.1, pp.~annurev--soc--073117--041351. ISSN: 0360-0572. DOI:
10.1146/annurev-soc-073117-041351.
\url{https://www.annualreviews.org/doi/10.1146/annurev-soc-073117-041351}.

Elo, Irma T, Cassio M. Turra, Bert. Kestenbaum, and B. Renee. Ferguson
(2004). ``Mortality Among Elderly Hispanics in the United States: Past
Evidence and New Results''. In: \emph{Demography} 41.1, pp.~109--128.
ISSN: 1533-7790. DOI: 10.1353/dem.2004.0001.
\url{http://link.springer.com/10.1353/dem.2004.0001}.

Hayward, Mark D, Robert A. Hummer, Chi-Tsun Chiu, César
González-González, and Rebeca Wong (2014). ``Does the Hispanic Paradox
in U.S. Adult Mortality Extend to Disability?''. In:
\emph{Population Research and Policy Review} 33.1, pp.~81--96. ISSN:
0167-5923. DOI: 10.1007/s11113-013-9312-7.
\url{http://link.springer.com/10.1007/s11113-013-9312-7}.

Mehta, Neil K, Irma T. Elo, Michal Engelman, Diane S. Lauderdale, and
Bert M. Kestenbaum (2016). ``Life Expectancy Among U.S.-born and
Foreign-born Older Adults in the United States: Estimates From Linked
Social Security and Medicare Data''. In: \emph{Demography} 53.4,
pp.~1109--1134. ISSN: 0070-3370. DOI: 10.1007/s13524-016-0488-4.
\url{http://link.springer.com/10.1007/s13524-016-0488-4}.

Riosmena, Fernando, Rebeca Wong, and Alberto Palloni (2013). ``Migration
Selection, Protection, and Acculturation in Health: A Binational
Perspective on Older Adults''. In: \emph{Demography} 50.3,
pp.~1039--1064. ISSN: 0070-3370. DOI: 10.1007/s13524-012-0178-9.
\url{http://link.springer.com/10.1007/s13524-012-0178-9}.

\hypertarget{week-5-intergroup-relations}{%
\subsection{Week 5: Intergroup
Relations}\label{week-5-intergroup-relations}}

Crowley, Martha and Daniel T. Lichter (2010). ``Social Disorganization
in New Latino Destinations?*''. In: \emph{Rural Sociology} 74.4,
pp.~573--604. ISSN: 00360112. DOI: 10.1111/j.1549-0831.2009.tb00705.x.
\url{http://doi.wiley.com/10.1111/j.1549-0831.2009.tb00705.x}.

Hall, Matthew and Jacob Hibel (2017). ``Latino Students and White
Migration from School Districts, 1980-2010''. In: \emph{Social Problems}
64.4, pp.~457--475. ISSN: 0037-7791. DOI: 10.1093/socpro/spx029.
\url{http://academic.oup.com/socpro/article/64/4/457/4259432/Latino-Students-and-White-Migration-from-School}.

Lee, Barrett A, Michael J.R. Martin, and Matthew Hall (2017).
``Solamente Mexicanos? Patterns and sources of Hispanic diversity in
U.S. metropolitan areas''. In: \emph{Social Science Research} 68,
pp.~117--131. ISSN: 0049-089X. DOI: 10.1016/J.SSRESEARCH.2017.08.006.
\url{https://www.sciencedirect.com/science/article/pii/S0049089X16305488}.

Lichter, Daniel T. (2012). ``Immigration and the New Racial Diversity in
Rural America''. In: \emph{Rural Sociology} 77.1, pp.~3--35. DOI:
10.1111/j.1549-0831.2012.00070.x.
\url{http://doi.wiley.com/10.1111/j.1549-0831.2012.00070.x}.

Lichter, Daniel T, Domenico Parisi, Michael C. Taquino, and Steven
Michael Grice (2010). ``Residential segregation in new Hispanic
destinations: Cities, suburbs, and rural communities compared''. In:
\emph{Social Science Research} 39.2, pp.~215--230. ISSN: 0049-089X. DOI:
10.1016/J.SSRESEARCH.2009.08.006.
\url{https://www.sciencedirect.com/science/article/pii/S0049089X09000908}.

\hypertarget{week-6-economics-mobility-and-migration-in-the-united-states}{%
\subsection{Week 6: Economics, Mobility, and Migration in the United
States}\label{week-6-economics-mobility-and-migration-in-the-united-states}}

Calnan, Ray and Gary Painter (2017). ``The response of Latino immigrants
to the Great Recession: Occupational and residential (im)mobility''. In:
\emph{Urban Studies} 54.11, pp.~2561--2591. ISSN: 0042-0980. DOI:
10.1177/0042098016650567.
\url{http://journals.sagepub.com/doi/10.1177/0042098016650567}.

Golash-Boza, Tanya (2009). ``A Confluence of Interests in Immigration
Enforcement: How Politicians, the Media, and Corporations Profit from
Immigration Policies Destined to Fail''. In: \emph{Sociology Compass}
3.2, pp.~283--294. ISSN: 17519020. DOI:
10.1111/j.1751-9020.2008.00192.x.
\url{http://doi.wiley.com/10.1111/j.1751-9020.2008.00192.x}.

Hall, Matthew, Emily Greenman, and Youngmin Yi (2019). ``Job Mobility
among Unauthorized Immigrant Workers''. In: \emph{Social Forces} 97.3,
pp.~999--1028. ISSN: 0037-7732. DOI: 10.1093/sf/soy086.
\url{https://academic.oup.com/sf/article/97/3/999/5078444}.

Parrado, Emilio A. and Chenoa A. Flippen (2016). ``The Departed:
Deportations and Out-Migration among Latino Immigrants in North Carolina
after the Great Recession''. In:
\emph{The ANNALS of the American Academy of Political and Social Science}
666.1. Ed. by Katharine M. Donato and Douglas S. Massey, pp.~131--147.
DOI: 10.1177/0002716216646563.
\url{http://journals.sagepub.com/doi/10.1177/0002716216646563}.

\hypertarget{week-7-transnational-communities}{%
\subsection{Week 7: Transnational
Communities}\label{week-7-transnational-communities}}

Anguiano-Tellez, Maria Eugenia and Alma Paola Trejo-Pena (2007).
``Vigilante and control at the U.S.-Mexico border region: The new routes
of international flows.''. In: \emph{Papeles de Poblacion} 13.51,
pp.~45--75.

Durand, Jorge (1986). ``Circuitos migratorios en el occidente de
Mexico''. In: \emph{Revue européenne des migrations internationales}
2.2, pp.~49--67. ISSN: 0765-0752. DOI: 10.3406/remi.1986.1098.
\url{https://www.persee.fr/doc/remi{\_}0765-0752{\_}1986{\_}num{\_}2{\_}2{\_}1098}.

GUARNIZO, LUIS E. (1994). ``Los Dominicanyorks:''. In:
\emph{The ANNALS of the American Academy of Political and Social Science}
533.1, pp.~70--86. ISSN: 0002-7162. DOI: 10.1177/0002716294533001005.
\url{http://journals.sagepub.com/doi/10.1177/0002716294533001005}.

Levitt, Peggy and Nina Glick Schiller (2006). ``Conceptualizing
Simultaneity: A Transnational Social Field Perspective on Society1''.
In: \emph{International Migration Review} 38.3, pp.~1002--1039. ISSN:
01979183. DOI: 10.1111/j.1747-7379.2004.tb00227.x.
\url{http://doi.wiley.com/10.1111/j.1747-7379.2004.tb00227.x}.

Tarrius, Alain (2000). ``Leer, describir, interpretar las circulaciones
migratorias: conveniencia de la noción de ``territorio circulatorio''.
Los nuevos hábitos de la identidad''. In: \emph{Relaciones} 21.83,
pp.~37--66.
\url{https://biblat.unam.mx/es/revista/relaciones-colmich-zamora/articulo/leer-describir-interpretar-las-circulaciones-migratorias-conveniencia-de-la-nocion-de-territorio-circulatorio-los-nuevos-habitos-de-la-identidad}.




\end{document}

\makeatletter
\def\@maketitle{%
  \newpage
%  \null
%  \vskip 2em%
%  \begin{center}%
  \let \footnote \thanks
    {\fontsize{18}{20}\selectfont\raggedright  \setlength{\parindent}{0pt} \@title \par}%
}
%\fi
\makeatother
